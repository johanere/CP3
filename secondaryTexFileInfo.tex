% ****** Start of file aipsamp.tex ******
%
%   This file is part of the AIP files in the AIP distribution for REVTeX 4.
%   Version 4.1 of REVTeX, October 2009
%
%   Copyright (c) 2009 American Institute of Physics.
%
%   See the AIP README file for restrictions and more information.
%
% TeX'ing this file requires that you have AMS-LaTeX 2.0 installed
% as well as the rest of the prerequisites for REVTeX 4.1
%
% It also requires running BibTeX. The commands are as follows:
%
%  1)  latex  aipsamp
%  2)  bibtex aipsamp
%  3)  latex  aipsamp
%  4)  latex  aipsamp
%
% Use this file as a source of example code for your aip document.
% Use the file aiptemplate.tex as a template for your document.
\documentclass[aip,nobalancelastpage,
twocolumn,
%sd,%
rsi,%
 amsmath,amssymb,
%preprint,%
 reprint,%
%author-year,%
%author-numerical,%
]{revtex4}

\usepackage{float}
\usepackage{amsfonts}
\usepackage[]{algorithmic}
\usepackage{graphicx}% Include figure files
\usepackage{dcolumn}% Align table columns on decimal point
\usepackage{listings}
\usepackage[utf8]{inputenc}
\usepackage[norsk]{babel}
\usepackage{bm}% bold math
\usepackage{verbatim}
\usepackage{hyperref}
\usepackage{xcolor}

%\usepackage[mathlines]{lineno}% Enable numbering of text and display math
%\linenumbers\relax % Commence numbering lines



\begin{document}

\preprint{AIP/123-QED}

\title{Project 3: Solar System}

\author{Noah Oldfield}

\affiliation{ 
Department of physics, University of Oslo%\\This line break forced with \textbackslash\textbackslash
}%

\date{\today}% It is always \today, today,
             %  but any date may be explicitly specified

\begin{abstract}


\end{abstract}

\keywords{Suggested keywords}%Use showkeys class option if keyword
                              %display desired
\maketitle

Github repository: \url{https://github.com/nhofield/fys3150Projects.git} 


\section{Theory}

\subsection{Conservation of Angular momentum, potential and kinetic energy}
\subsubsection{Circular orbit, two body system}
For a two-body system consisting of objects with masses $m_1$ and $m_2$ there is a gravitational force acting on $m_1$ from $m_2$ given by newtons gravitational law. Also newtons third law gives the force from $m_1$ on $m_2$ as being equal in magnitude, but with opposite sign.\par
The torque $\vec{\tau}_1$ on $m_1$ and $m_2$ is
\begin{align}
\vec{\tau}_1 = \vec{r} \times \vec{F}_1 = 0 \\
\vec{\tau}_2 = \vec{r} \times \vec{F}_2 = 0 \\
\end{align}
since the distance $\vec{r}$ and force $\vec{F}$ are parallell.
This gives
\begin{equation}
\frac{d \vec{L}}{dt} = \sum_j \vec{\tau} = 0 \longrightarrow \vec{L} = constant
\end{equation}
\subsubsection{Elliptical orbits and total angular momentum}
Consider now a system of $n$ bodies of masses $m_1,m_2,\cdots,m_n$. For each mass, for example $m_1$ there is a force $\vec{F}_1$ and distance $\vec{r}_1$ for each coupling with another mass. This means that there is a sum of torques on each object in the system which each are zero. Again leading to conservation of angular momentum for each body. Now take the total angular momentum of the system, this will be constant since the total angular momentum is a sum of constant angular momentums.

\subsection{Escape Velocity}
The escape velocity is defined as the minimum velocity, required of a celestial body, in order to escape the gravitational field of star system. Considering a case of the two-body solar system with earth. The escape velocity of earth $v_{esc}$, is then given by the condition where the earth's gravitational potential and kinetic energy are equal. Providing the formula
\begin{equation}
v_{esc} = \sqrt{\frac{2 G M_E}{r}}
\end{equation}




\section{Results}
\subsection{Two-Body system non object oriented}
For $N=10^5$, $h=^{-5}$, orbital period of 1 year..
Angular momentum, potential energy and kinetic energy conserved within tolerances $\epsilon_{Verlet}$ and $\epsilon_{Euler}$
\begin{align*}
\epsilon_{Verlet} &= 10^{-12} & \epsilon_{Euler} = 10^{-7}
\end{align*}

\end{document}


